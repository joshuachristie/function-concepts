
\documentclass{article}
\usepackage{graphicx}
\usepackage{threeparttable}
\usepackage{amssymb}

\title{Concepts of function in biology and biomedicine}

\begin{document}
\maketitle

\section{Abstract}
\label{sec:abstract}

Several philosophical accounts of disease are constructed at least partly around an objective biological criterion. Under these accounts, we can define disease as the failure of physiological parts or processes to perform their proper function or an ``impairment of normal functioning''. Determining whether a phenotype—such as obesity—is a disease or determining the level of functioning at which some aspect of physiology—such as response to insulin—becomes pathological throws considerable weight on the concept of biological function. However, there are a number of philosophical theories of function, each of which defines function differently. It is not clear which theory—or combination of theories—we should use to explicate the medical conception of function. One reason for this is that we have no systematic way to determine how biologists and medical practitioners conceive of, or write about, function in their respective disciplines. To further complicate matters, natural language is replete with ambiguities, and scientific manuscripts often use technical terms imprecisely. Without a descriptive understanding of how different conceptions of function are used in biology and medicine, we have little hope of bringing insights on biological function to bear on disputes about function and malfunction in medicine. Here we develop a systematic method for analysing biological function by outlining a classification scheme that combines syntactic and semantic analysis in a dependency-grammar framework.


\section{Introduction}
\label{sec:introduction}

Much of philosophy of medicine is concerned with defining issues surrounding the concept of disease.
Philosophical approaches to categorising disease can be broadly divided into two camps: a constructivist view that prioritises social value judgements and a naturalist view that prioritises biological theory.
The former contends that disease should be primarily based on whether or not society disvalues the condition in question, whereas the latter holds that our categorisation of disease should be primarily based on whether ``something has gone wrong'' in a biological system.
But what does it mean for ``something to go wrong'' with a biological system according to the naturalist view?

Before determining what it means for ``something to go wrong'', one requires a normative framework that can be used to determine what a biological component (trait) ought to do.
To ground their normative claims in biology, many naturalist accounts defer to the notion of function.
If a trait's function is what it ought to do, then it is a reasonable proposition that dysfunction of the trait leads to disease.
But herein lies a several problems.
There are multiple philosophical theories of function, each differing in their claims to normativity (and some with arguably no normative claims whatsoever).
Furthermore, the word function is used in a variety of ways by practitioners---biologists, medical researchers, and physicians---and it is rarely clear which concept of function is being used in any given circumstance or how well real-world usage maps onto philosphical conceptions.
One cannot easily make progress on a theory without first setting a solid foundation.
If we hope to use experimental philosophy to understand a naturalistic account of disease and dysfunction, we must first clarify how the concept of function is used in practice.






\section{Introduction-old-draft}
\label{sec:introduction-1}

-ENCODE and the debate
-Most biologists just talk of CR and SE but there are other types (activity (note Neander 2017 calls this minimal function), advantage, function-dysfunction)

describe the functions briefly but also include token type distinction and show how they're related (will need to reorganise a bit). determinable property (e.g. gene) and determinate property (e.g. allele) are both type-level properties. A particular allele in an individual would be a token-level property. (actually think the descriptions can go in the four questions bit. would be good to just start off with a high level description with some history/references)

Causal role function tends to talk of characters (or items or behaviours) and is interested in how they work within a complex system. It's mainly concerned with determinable types, as we want to talk about what this part or component does within a system. (Though I might need to do a little reading on this because Garson 2019 says that CR people consider a CR token being dysfunctional as being the same as having no function. I suppose CR need not be applied to types, as it doesn't have a lineage concept, but nevertheless it seems to be applied to types. For example, if a biologist runs an experiment, they use many individuals because they want to get at the type characterisation not token.)

Biological advantage function is causal role function with two differences: (i) it's comparative (trait $i$ vs trait $j$) and (ii) it's interested in variants of types (traits) rather than the characters themselves. BA functions answer the question ``what advantage does this trait variant give its bearer over another (counterfactual or actual) organism with a different trait variant?''.

Selected effects functions deal again with advantages of trait variants, only now it asks whether these advantages have accumulated in such a way that we can tell a causal story about how the effect of the trait variant has led to the evolution of the trait. A criticism of SE is that it doesn't tell you anything about the current function of the trait (see 2009 organizational closure paper), which is another argument for pluralism.

The functional-dysfunctional question deals with tokens---how has the trait evolved in the development of a particular individual? Although it must be said that this isn't the cleanest distinction. While we'll want to ascribe a dysfunctional state (e.g. disease) to an individual, we might cluster underlying etiologies into a single coarse grain phenotype (e.g. a syndrome). Might also cluster a single genotype change (e.g. mito diseases). This could also be without regard for trait (e.g. some mito diseases will apply to the character (all human mito haplotypes), though need to check that this is true), but perhaps some are specific to some traits.
Note that I think part of this comes down to the fact that it's not well-defined whether a new token is a new trait or whether it is a dysfunctional member of the type (i.e. are we dealing with a collection of dysfunctional tokens who are grouped based on either a common inheritance, common architecture, common phenotype, or a new trait type? (you could go via inheritance vs ontogeny but there might be some diseases that are hard to tease apart here? e.g. mito disease could be a mutation from mother, stable inheritance, mutation in zygote, etc.---would we really distinguish them if they're the same structurally? I need to be careful here because I'm claiming that dysfunction is ontogenic so I really need to be able to make the claim that dysfunction is token-level which means syndromes/diseases are collections of token dysfunctions. Note Neander 2017 says that malfunctions are token level.)
One way of talking about when a dysfunction becomes a trait is when it is heritable...this will lead to some odd cases (mitochondrial diseases becoming traits), or we just accept that harmful dysfunctions that are heritable are a grey zone. Huntington's disease is another good example because it can be caused by ``slippage'' of genetic repeats so is ontogenic in this sense, yet it is also heritable.
Neander (2017) actually seems to want to say that ``not performing function = malfunction'' due to mismatch (pg 1152)

EDIT 20201015

I want to split up the current utility case into a positive aspect (biological advantage) and a negative case (basically making the claim that because current utility contrasts traits, one can also talk about a biological disadvantage). Now of course for the negative case it doesn't provide an explanation for why something is there at least not in the biological advantage sense (it could be a SE that's no longer adaptive though). To fully explore the possibilities, I think we need to allow for biological disadvantage though (e.g. something that was selected for in the past but is now actively being selected against...e.g. a mismatch or something to do with the genetic architecture or development of the trait). While I already had this in the previous scheme, what's different now is the following. For \emph{some} of these cases of traits at a biological disadvantage, we will want to say that the traits are dysfunctional and that the carriers are diseased (e.g. sickle cell). This a way to get dysfunction at the trait (not token) level.

A second alteration is to let the functional-dysfunctional axis to be hyperfunctional-functional-dysfunctional. An individual's trait can perform its function (``functional''), it can fail to perform its function (``dysfunctional'') or it can perform its function better than expected (``hyperfunctional''). The latter might be due to plasticity (or even just random variation, such as with a ``lucky'' quantitative trait), or it might be due to a beneficial mutation, in which case it might lead to a new trait that will, in the future, be considered the wild type. Although I suppose that the latter case is technically not actually ontogeny (since the mutation occurs pre-fertilization), I chatted to Paul about this and he seems to think it would be fine to consider pre/post fertilization mutations as equivalent.

Will also need to address organizational theories and their claim of normativity. Will need to sidestep all the details but basically the main point is for dysfunction you need two things: (i) a Cummins-style functional analysis but of how an item is working in a token (not character); and (ii) a normative grounding that tells you how it ``ought to work''. All approaches must borrow elements from each other to make this work (SE needs CR; CR needs SE or some other framework that ascribes type-level normativity; organizational must borrow elements from CR and SE (for the latter need to see whether they've developed the reproduction aspect). So the point is that we need pluralism of the three function questions to get at dysfunction.

(This actually works well because it provides a nice justification for separating out the first three (types) for one framework and laying the last (token) over this framework.

also point out how SE function can generally only be established once we have ideas about CR and BA. The first thing we ask is ``what does this trait do?'' or ``how is this trait used?'' (actually these seem like the second thing; first thing should be ``how does it work?''); the second thing is ``what benefit does it provide to the organism?''; the third thing is ``might it have provided a benefit to the organism in the past?''; the fourth thing is ``can we explicitly show how the trait could have evolved due to its function(al effects)?''.
Although one might get away with just generically knowing what it does and hypothesizing an advantage (or even just knowing the advantage), without knowing details about how it works (e.g. genetic architecture, mechanism of action), it will be hard to build accurate models.
So with this I can basically deal with the issue I wanted to raise about how some (e.g. Garson) think that SE has greater explanatory power because they are, in part, conflating what we know because we know SE vs knowing all 3 (although obviously knowing SE is extremely important if you want to ask an evolutionary question)

another thing I can show is why many fields don't need to take an evolutionary approach to make accurate inferences, especially things to do with disease. I suspect here I can focus on the fact that disease is on the functional-dysfunctional continuum, which relates to how well a \emph{ahistorical} notion of function is working (i.e. CR or BA). It only relates to dysfunction of SE in the sense of dysfunction of a CR/BA doing what it was selected for (but even here the dysfunction relates specifically to the ahistorical CR function. So it follows that if we are interested in how something works or why somebody is diseased, then we don't necessarily need to know the evolutionary history (except how it relates to the present). It's also important to distinguish between SE dysfunction and the others. SE, being historical, can't dysfunction in the sense that it changes performance in a token whereas CR/BA dysfunction could (a CR instance having no function will have a similar phenotypic effect in a token to a CR instance having a function but being dysfunctional). If we define SE dysfunction as ``not being able to perform the CR for which it was selected for constitutional reasons'', then SE dysfunction is a broken CR mechanism in a selective/Normal environment. This is similar to what one might call a CR dysfunction, but note that garson would say that CR can't dysfunction because if a CR doesn't work then it doesn't have a function. I think I'll need to sidestep these issues and just focus on the fact that everybody would generally agree that dysfunction is when the trait is not ``working as it ought to'' with the ought to part being the source of most of the controversy.

note that the above case becomes much stronger if you subscribe to harmful dysfunction, as this puts even less emphasis on SE normativity.

Will need a section that looks at implicit use of other forms of function (e.g. CR functions assuming it leads to an advantage, BA functions assuming it will be selected for, SE functions taking for granting that CR/BA functions have been established, etc.)

Much argument seems to revolve around which conception of function is best. I will have shown that this misses the point: biologists and medical researchers want to be able to ask four function questions and no conception of function alone can currently serve this purpose.

mention that people might talk of something functioning in a disease but this is not the function of the thing (think Doolittle also made this point)

reason we want to know ontogeny as part of the four is that it would help us distinguish between an organism with a strong current utility but mildly dysfunctional trait and an organism with a weak current utility but fully functional trait

maybe point out that it's not useful to think of tinbergen's questions as two historical/two ahistorical (because although development is historical, it's over a timescale that is arguably shorter than the ahistorical notions of mechanism/utility) (probably just ignore this if possible though)

Character can be used to refer to a range of different biological items: a genetic sequence,
(Actually this is interesting. Technically a character should be a phenotype but then can we not talk of a gene as a character? Seems like if we can make a fairly straightforward genotype->phenotype mapping then we can do this, but if isn't? I'm trying to say that causal role functions are character-level but they need not be...we could just look at a transcription factor or something. So I guess the key point to make is that causal role functions need not be character level but they can be and are often employed in this manner.

causal/biological role---\textbf{mechanism}
biological advantage---\textbf{current utility}? (Nesse didn't like this but I think that's because he totally misunderstands what it actually is. I think it's fine because I agree that adaptive value will be misinterpreted more easily and Nesse is a case in point. Note that bekoff1995 used this term so it wasn't introduced by the 2013 tree paper)
selected effects---\textbf{evolution}
functional--dysfunctional---\textbf{ontogeny}

Function is used in a multitude of ways in the biological and medicine sciences.
There has been renewed interest in how function is used biology, in part due to debate over the ENCODE consortium and their annotation of functional elements of the human genome.
Several evolutionary biologists have weighed in on the debate (cite Doolittle, Graur), introducing some of the philosophical notions of biological function.
These have focused on two contrasting conceptions in function in particular: causal role and selected effects.
Causal role functions are utilised widely in fields like physiology, molecular biology, and genomics.
Causal role explanations view function as the mechanistic role that an item plays in a biological system.
Selected effects functions are primarily utilised in evolutionary biology (behavioural ecology?, some genetics?).
Selected effects explanations view functions as those effects of a trait that were selected for by natural selection.
Under this framework, a trait's function(s) is taken to constitute an explanation for the existence of the trait.
Although causal role and selected effects functions are the two concepts that have received the most attention from philosophers, there are a number of other notions.
In this paper, I show that the questions that philosophers address with their frameworks, and biologists address with their usage, can be cast into the familiar mould of Tinbergen's four questions.

\section{Function and the four questions}
\label{sec:funct-tinb-four}

Before I introduce function in the context of Tinbergen's four questions, I must first define some terminology, as philosophers and biologists often use quite different terms to refer to similar concepts.
One important distinction will be between what philosophers call \emph{types} and \emph{tokens}.
A type is a class of objects whereas a token is an instantiation of the class/type in an individual.
For example, the class ``chair'' is a type, but \emph{my} office chair is a token.
Philosophers further divide types into \emph{determinates} and \emph{determinables}.
A determinable property is a category that can be made more specific; a determinate property is a specific member of a determinable category. 
For example, the determinable property ``colour'' can be divided into the determinate properties ``red'' or ``blue''.
Now to translate these philosophical notions to biological terminology.
I will refer to a determinable (type) property of an organism as a \emph{character}, a determinate (type) property of an organism as a \emph{trait}, and a token property of a particular organism as a \emph{token trait}.
A character either refers to a phenotype or to an item that displays a phenotype, which includes behaviours, morphological features, production of biomolecules such as proteins and hormones, or even a genetic sequence (if there is a simple mapping to a phenotype).
A trait is a specific variant of a character, and a token trait is an instance of a trait in an individual.
For example, eye colour is a character, blue eyes is a trait, and \emph{my} blue eyes are a token trait.

\subsection{Mechanism: how does it work?}
\label{sec:mechanism}

A mechanistic analysis is applied to \emph{characters}.
One of the first questions we might ask of a character is ``how does it work?'' or ``what role does it play in an organism (or a system of the organism)?''.

We are generally not concerned here with traits.

Tinbergen denoted this question as ``causation'', but modern interpretations have preferred mechanism (cite bateson/laland).
This question is the domain of functional analysis, the name given to the approach from which we get causal role functions (cummins1975).
%Functional analysis, and therefore causal role functions, are closely intertwined with a reductionist approach to the scientific method.
It works roughly as follows.
An investigator chooses a complex biological system that has an interesting capacity (in the eyes of the investigator).
In the simplest case, she may analyse the system by reference to how its parts interact to produce said capacity.
If the system is particularly complex, she may break it down into various subsystems, and analyse the capacity of each of those subsystems with respect to how its parts interact.
This process can be recursively applied if the investigator wants to simplify the system further.
For example, let us say that the investigator is interested in how the cardiovascular system (system of interest) circulates blood and transports nutrients (capacity) in humans.
She can divide the cardiovascular system into subsystems (heart, lungs, blood vessels, etc.), each of which itself has a capacity and various parts.
Let's say she's interested how the heart (subsystem) pumps blood (capacity of subsystem).
Specifically, she wants to study how the heart's electrical conduction system's (subsubsystem) ability to regulate heart beat (capacity of subsubsystem) is affected by the Purkinje fibers's (subsubsubsystem) ability to conduct action potentials (capacity of a subsubsubsystem). (Should actually only go up to subsubsystem and then later show how you might break it down even more and design an experiment around it. Also need to be clearer on what is a system and what is a component...is there a difference? Finally, need to actually show how causal role functions assign functions---i.e. what are the functions in these examples---and give a compact definition.)
After doing so, she works her way back, taking the insights about how the Purkinje fibers affect the heart's electrical conduction system to better understand the heart.
Her better understanding of the heart in turn improves her understanding of the cardiovascular system ability to circulate blood and transport nutrients.
Although the terminology is laborious, this approach is consistent with how much biology is conducted and goes by many names: reductionism, mechanistic approach, or bottom-up approach.
Even a field like systems biology, which rejects the reductionist label, mechanistically model interactions between components of systems. (need to read up on this)



\subsection{Current utility: how is it used?}

Tinbergen denoted the second question as ``survival value'', but this term has fallen out of fashion because it highlights but one part of an organism's biological fitness.
I adopt the terminology of bateson/laland and call this current utility.

note that this is often called adaptive in biology

\subsection{Evolution: how did it evolve?}
\label{sec:evolution:-how-did}

note the connection with adaptation

\subsection{Ontogeny: how did it develop?}
\label{sec:ontogeny:-how-did}



\section{Relations between the functions}
\label{sec:relat-betw-diff}

\begin{figure}[ht]
  \centering
  \includegraphics[width=\linewidth]{venn_function.pdf}
  \caption[\textbf{Relationship between causal role, biological advantage, and selected effects functions.}
]{Biological advantage functions (BA) are the subset of causal role functions (CR) that affect organism fitness in the present time ($t$). For a function to become a selected effect function (SE) at a future time ($t'$), it must have been a biological advantage function. Not all current biological advantage functions, however, will become selected effects functions.
}
\label{fig:venn}
\end{figure}

\section{Function pluralism: asking the four questions}
\label{sec:funct-plur-asking}


\begin{table}
\centering
\begin{threeparttable}
\caption{Combining ``type'' notions of function to classify traits (note that this does not include the dimension of dysfunction)}

\begin{tabular}{cccccc}
  \multicolumn{3}{c}{Function categories}&
  \multicolumn{1}{c}{}&
\multicolumn{2}{c}{Biological description}\cr
\cline{1-3}
\cline{5-6}
CR & BA & SE && Classification& Examples \\
\hline

\checkmark & \checkmark & \checkmark && Paradigmatic function & Heart pumping blood \\
\checkmark & \checkmark & $\times$ && Exaptation & Feathers for flying \\
\checkmark & $\times$ & \checkmark && Mismatch & Fat/sugar and obesity \\
\checkmark & $\times$ & $\times$ && Nearly-neutral traits & biochemicaly-active junk DNA \\
$\times$ & $\times$ & \checkmark && Vestigial & Human appendix,flightless bird wings, pseudogenes \\
$\times$ & $\times$ & $\times$ && Spandrels/byproduct/parasitic DNA\tnote{1} & Transposable elements \\

\hline

\end{tabular}
\begin{tablenotes}
\item[1] Anything that was ``selected of'' rather than ``selected for'' (spandrels, byproducts, constraints) or something that was forced on an organism (e.g. viral DNA or transposable elements). The former evolved for other reasons and may not do anything for the organism while the latter primarily ``serves'' as a selfish element that is using the organism to spread its DNA.

\end{tablenotes}

\end{threeparttable}
\end{table}

\section{Conclusion}
\label{sec:conclusion}

Note that biologists sometimes confuse a weaker notion of mechanism with causal role.
ENCODE's definition of function, for example (and Grauer, etc mischaracterisation of CR)
(This isn't straightforward, because ENCODE's definition doesn't really deal with characters. I might just need to deal with this at the end.)


\end{document}
%%% Local Variables:
%%% mode: latex
%%% TeX-master: t
%%% End:
