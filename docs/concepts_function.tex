\documentclass{article}

\title{Concepts of function in biology and biomedicine}

\begin{document}
\maketitle

\section{Abstract}
\label{sec:abstract}

Several philosophical accounts of disease are constructed at least partly around an objective biological criterion. Under these accounts, we can define disease as the failure of physiological parts or processes to perform their proper function or an ‘impairment of normal functioning’. Determining whether a phenotype—such as obesity—is a disease or determining the level of functioning at which some aspect of physiology—such as response to insulin—becomes pathological throws considerable weight on the concept of biological function. However, there are a number of philosophical theories of function, each of which defines function differently. It is not clear which theory—or combination of theories—we should use to explicate the medical conception of function. One reason for this is that we have no systematic way to determine how biologists and medical practitioners conceive of, or write about, function in their respective disciplines. To further complicate matters, natural language is replete with ambiguities, and scientific manuscripts often use technical terms imprecisely. Without a descriptive understanding of how different conceptions of function are used in biology and medicine, we have little hope of bringing insights on biological function to bear on disputes about function and malfunction in medicine. Here we develop a systematic method for analysing biological function by outlining a classification scheme that combines syntactic and semantic analysis in a dependency-grammar framework.

\section{Introduction}
\label{sec:introduction}

\end{document}
%%% Local Variables:
%%% mode: latex
%%% TeX-master: t
%%% End:
